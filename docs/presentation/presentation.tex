\documentclass{beamer}
\usepackage{graphicx}
\usepackage{xcolor}
\usepackage{listings}
\usepackage{fontawesome5}

% Theme and Colors
\usetheme{Madrid}
\usecolortheme{default}
\setbeamertemplate{navigation symbols}{}
\setbeamertemplate{footline}[page number]

% Define colors for code
\definecolor{codegreen}{rgb}{0,0.6,0}
\definecolor{codegray}{rgb}{0.5,0.5,0.5}
\definecolor{codepurple}{rgb}{0.58,0,0.82}
\definecolor{backcolour}{rgb}{0.95,0.95,0.92}

\lstset{
    backgroundcolor=\color{backcolour},
    commentstyle=\color{codegray},
    keywordstyle=\color{codepurple},
    numberstyle=\tiny\color{codegray},
    stringstyle=\color{codegreen},
    basicstyle=\ttfamily\small,
    breakatwhitespace=false,
    breaklines=true,
    captionpos=b,
    keepspaces=true,
    numbers=left,
    numbersep=5pt,
    showspaces=false,
    showstringspaces=false,
    showtabs=false,
    tabsize=2
}

% Title Page
\title{Modern Software Development}
\subtitle{A Simple Guide to Today's Technology}
\author{Technical Concepts Explained Simply}
\date{\today}

\begin{document}

\frame{\titlepage}

% Table of Contents
\begin{frame}
\frametitle{What We'll Learn Today}
\begin{itemize}
    \item \textbf{Docker:} Shipping software like shipping boxes
    \item \textbf{SSR (Server-Side Rendering):} Where should your code run?
    \item \textbf{SvelteKit:} Modern web framework
    \item \textbf{Go vs Python:} Different approaches to writing programs
\end{itemize}
\end{frame}

% ============================================================================
% SECTION 1: DOCKER
% ============================================================================
\section{Docker: Containers Explained}

\begin{frame}
\frametitle{Think of Docker Like Shipping}
\begin{center}
    \Large \textbf{Traditional Shipping Problem}
\end{center}

Before containers, shipping goods was messy:
\begin{itemize}
    \item Books packed differently than clothes
    \item Different sizes required different trucks
    \item Items could get damaged during transport
    \item No standard way to pack things
\end{itemize}

\vspace{1cm}

\begin{center}
    \textbf{Solution:} The shipping container (standardized box)
    \begin{itemize}
        \item All items fit in same sized box
        \item Easy to stack and transport
        \item Protected from damage
        \item Works on ships, trucks, trains
    \end{itemize}
\end{center}
\end{frame}

\begin{frame}
\frametitle{Docker Works the Same Way}

\textbf{Docker Container} = A box that contains everything your software needs:
\begin{itemize}
    \item Your code
    \item Required libraries
    \item Operating system basics
    \item Configuration files
\end{itemize}

\vspace{1cm}

\textbf{Why this matters:}
\begin{itemize}
    \item \textcolor{green}{Works the same on your computer, my computer, or a server}
    \item \textcolor{green}{No more ``It works on my machine'' problems}
    \item \textcolor{green}{Easy to update and ship out}
\end{itemize}
\end{frame}

\begin{frame}
\frametitle{Docker: The Real-World Benefit}

\begin{columns}
\column{0.5\textwidth}
\textbf{WITHOUT Docker:}
\begin{itemize}
    \item Developer installs software locally
    \item Moves to production server
    \item Server has different setup
    \item Software breaks
    \item Hours spent debugging
\end{itemize}

\column{0.5\textwidth}
\textbf{WITH Docker:}
\begin{itemize}
    \item Developer creates container
    \item Container works everywhere
    \item Server runs exact same container
    \item Everything works
    \item Deploy in minutes
\end{itemize}
\end{columns}

\vspace{1cm}

\textbf{Think of it:} Like mailing a self-contained lunch box that works anywhere, instead of just mailing ingredients.
\end{frame}

\begin{frame}
\frametitle{Docker Vocabulary Made Simple}

\begin{description}
    \item[\textbf{Image}] A template or blueprint (like a recipe)
    \item[\textbf{Container}] A running copy of that template (like the cooked meal)
    \item[\textbf{Dockerfile}] Instructions for building the image (the recipe itself)
    \item[\textbf{Docker Hub}] Library of pre-made images (cookbook)
\end{description}

\vspace{1cm}

\textbf{Analogy:}
\begin{itemize}
    \item Image = Recipe
    \item Container = Meal you cooked from that recipe
    \item Dockerfile = How you write the recipe
    \item Docker Hub = Book of recipes
\end{itemize}
\end{frame}

% ============================================================================
% SECTION 2: SERVER-SIDE RENDERING (SSR)
% ============================================================================
\section{SSR: Where Should Code Run?}

\begin{frame}
\frametitle{The Server vs Browser Dilemma}

Modern websites need to decide: \textbf{Where does the work happen?}

\vspace{1cm}

\begin{columns}
\column{0.5\textwidth}
\textbf{Browser (Client)}
\begin{itemize}
    \item Your computer
    \item User's phone
    \item Software runs here
    \item Faster response
    \item More processing power needed on device
\end{itemize}

\column{0.5\textwidth}
\textbf{Server (Cloud)}
\begin{itemize}
    \item Powerful computer far away
    \item Software runs here
    \item Works on slow devices
    \item Requires internet
    \item Less power needed locally
\end{itemize}
\end{columns}
\end{frame}

\begin{frame}
\frametitle{Server-Side Rendering (SSR) Explained}

\textbf{What is SSR?}

The server does the heavy work BEFORE sending data to your browser.

\vspace{1cm}

\textbf{Real-World Analogy:}

\begin{description}
    \item[\textbf{Restaurant Kitchen (Server)}] Prepares your meal completely
    \item[\textbf{Your Dining Room (Browser)}] Receives a finished, hot meal
    \item[\textbf{Result}] Fast experience, ready to eat immediately
\end{description}

\vspace{1cm}

\textbf{Without SSR:}
\begin{itemize}
    \item Browser receives empty plate
    \item Browser must cook the meal
    \item Takes longer
    \item User waits more
\end{itemize}
\end{frame}

\begin{frame}
\frametitle{SSR Benefits for Different Users}

\textbf{Who benefits from SSR?}

\begin{itemize}
    \item \textcolor{green}{\textbf{Slow internet users:}} Less data to download
    \item \textcolor{green}{\textbf{Mobile users:}} Less work for phone battery
    \item \textcolor{green}{\textbf{Old devices:}} Don't need powerful hardware
    \item \textcolor{green}{\textbf{SEO (Google search):}} Better for search engines
\end{itemize}

\vspace{1cm}

\textbf{Why companies like SSR:}
\begin{itemize}
    \item Users see content faster
    \item Better search engine ranking
    \item Works better globally
    \item Better for all devices
\end{itemize}
\end{frame}

% ============================================================================
% SECTION 3: SVELTEKIT
% ============================================================================
\section{Modern Framework: SvelteKit}

\begin{frame}
\frametitle{What Are Web Frameworks?}

\textbf{A Framework:} A set of ready-made tools for building websites

\vspace{1cm}

\textbf{Think of it like building a house:}
\begin{itemize}
    \item Without framework: You make bricks, shape wood, mix cement
    \item With framework: Pre-made walls, doors, windows ready to assemble
\end{itemize}

\vspace{1cm}

\textbf{SvelteKit:}
\begin{itemize}
    \item Helps you build modern, fast websites
    \item Includes SSR capabilities
    \item Provides reusable components
    \item Handles routing (page navigation)
    \item Optimizes for users
\end{itemize}
\end{frame}

\begin{frame}
\frametitle{SvelteKit: The Simple Framework}

\textbf{SvelteKit's Philosophy:} Write less code, do more

\vspace{1cm}

\begin{columns}
\column{0.5\textwidth}
\textbf{What makes it simple:}
\begin{itemize}
    \item Smaller file sizes
    \item Less boilerplate
    \item Reactive by default
    \item Clean syntax
\end{itemize}

\column{0.5\textwidth}
\textbf{Best for:}
\begin{itemize}
    \item Quick prototypes
    \item Small teams
    \item Performance-critical apps
    \item Learning web dev
\end{itemize}
\end{columns}

\vspace{1cm}

\textbf{Analogy:} Like a small, nimble sailboat - fast and responsive
\end{frame}

% ============================================================================
% SECTION 4: GO LANGUAGE
% ============================================================================
\section{Go: The Language We Use}

\begin{frame}
\frametitle{What is Go?}

\textbf{Go:} A modern programming language designed for servers and APIs

\vspace{1cm}

\textbf{Go's Philosophy:} Simplicity, speed, and reliability

\vspace{1cm}

\textbf{Key characteristics:}
\begin{itemize}
    \item \textcolor{green}{\textbf{Super fast}} - Runs at incredible speeds
    \item \textcolor{green}{\textbf{Simple syntax}} - Easy to read and understand
    \item \textcolor{green}{\textbf{Compiled}} - Turns into fast machine code
    \item \textcolor{green}{\textbf{Perfect for servers}} - Handles thousands of users
    \item \textcolor{green}{\textbf{Small programs}} - Easy to package
\end{itemize}

\vspace{1cm}

\textbf{Why we chose Go:}
\begin{itemize}
    \item Perfect for building APIs (like our Fibers backend)
    \item Containerizes beautifully with Docker
    \item Extremely reliable and stable
\end{itemize}
\end{frame}

\begin{frame}
\frametitle{How Go Works: Compiled vs Interpreted}

\textbf{Three different approaches:}

\vspace{1cm}

\begin{description}
    \item[\textbf{Go (Compiled)}] 
    \begin{itemize}
        \item Converts code to machine language BEFORE running
        \item Like baking a cake completely before serving
        \item Super fast when it runs
        \item Must rebuild for each operating system
    \end{itemize}
    
    \vspace{0.5cm}
    
    \item[\textbf{Python (Interpreted)}]
    \begin{itemize}
        \item Reads and runs code line by line
        \item Like cooking while guests watch
        \item Slower execution
        \item Works on any computer without rebuilding
    \end{itemize}
    
    \vspace{0.5cm}
    
    \item[\textbf{Java (Compiled + Interpreted)}]
    \begin{itemize}
        \item Compiles to middle format
        \item Runs in special environment (JVM)
        \item Medium speed
        \item Complex setup
    \end{itemize}
\end{description}
\end{frame}

\begin{frame}
\frametitle{Go Compilation Process}

\textbf{How Go turns code into a fast program:}

\vspace{1cm}

\begin{enumerate}
    \item \textbf{You write Go code}
    \begin{itemize}
        \item Simple, readable syntax
    \end{itemize}
    \vspace{0.3cm}
    
    \item \textbf{Go compiler reads it}
    \begin{itemize}
        \item Checks for errors
        \item Optimizes the code
    \end{itemize}
    \vspace{0.3cm}
    
    \item \textbf{Converts to machine code}
    \begin{itemize}
        \item Native instructions for the CPU
        \item Specific to the operating system
    \end{itemize}
    \vspace{0.3cm}
    
    \item \textbf{Creates executable file}
    \begin{itemize}
        \item Standalone program
        \item No extra dependencies needed
        \item Ready to run
    \end{itemize}
\end{enumerate}

\vspace{1cm}

\textbf{Result:} Fast, reliable program that's easy to distribute
\end{frame}

\begin{frame}
\frametitle{Go vs Python: Speed Comparison}

\textbf{Why Go is faster:}

\vspace{1cm}

\begin{columns}
\column{0.5\textwidth}
\textbf{Go (Compiled)}
\begin{itemize}
    \item Code pre-processed
    \item Direct machine instructions
    \item Runs at CPU speed
    \item Minimal overhead
    \item \textcolor{green}{\textbf{VERY FAST}}
\end{itemize}

\column{0.5\textwidth}
\textbf{Python (Interpreted)}
\begin{itemize}
    \item Code read line by line
    \item Decisions made while running
    \item Translation overhead
    \item Extra interpretation layer
    \item \textcolor{orange}{\textbf{SLOWER}}
\end{itemize}
\end{columns}

\vspace{1cm}

\textbf{Real world example:}
\begin{itemize}
    \item Go handles 10,000 users: Still fast
    \item Python handles 1,000 users: Gets slow
    \item Go uses 50MB RAM: Light and lean
    \item Python uses 500MB RAM: Heavy
\end{itemize}
\end{frame}

\begin{frame}[fragile]
\frametitle{Understanding Composition: Go's Way}

\textbf{Go Example: Building a Dog}

\begin{lstlisting}[language=Go, caption=Go Composition Example]
type Animal struct {
    Name string
}

func (a Animal) Speak() string {
    return "Some sound"
}

type Dog struct {
    Animal  // Include Animal in Dog
    Breed string
}

// Override the Speak method
func (d Dog) Speak() string {
    return "Woof!"
}
\end{lstlisting}

\textbf{How it works:}
\begin{itemize}
    \item Dog ``contains'' an Animal
    \item Dog gets Animal's Name field
    \item Dog overrides the Speak method
    \item Simple and direct
\end{itemize}
\end{frame}

\begin{frame}
\frametitle{Go Advantages for Backend Development}

\textbf{Why Go is perfect for our backend:}

\vspace{1cm}

\begin{itemize}
    \item \textcolor{green}{\textbf{Lightning Speed}} - Compiled code runs super fast
    \item \textcolor{green}{\textbf{Low Memory}} - Uses minimal resources
    \item \textcolor{green}{\textbf{Easy Concurrency}} - Handles many users simultaneously
    \item \textcolor{green}{\textbf{Simple Syntax}} - Easy to write and understand
    \item \textcolor{green}{\textbf{Cross-Platform}} - Runs on Linux, Windows, Mac
    \item \textcolor{green}{\textbf{Docker Friendly}} - Creates tiny, fast containers
    \item \textcolor{green}{\textbf{Single Binary}} - One file to deploy
\end{itemize}

\vspace{1cm}

\textbf{Fibers Framework:}
\begin{itemize}
    \item Built in Go for maximum speed
    \item Perfect for building APIs
    \item Minimal overhead
\end{itemize}
\end{frame}

% ============================================================================
% SECTION 5: PUTTING IT TOGETHER
% ============================================================================
\section{Modern Development Stack}

\begin{frame}
\frametitle{How Everything Works Together}

\textbf{A typical modern web application:}

\vspace{1cm}

\begin{enumerate}
    \item \textbf{Backend (Server):} Written in Go or Python
    \begin{itemize}
        \item Handles business logic
        \item Stores data safely
        \item Runs in Docker containers
    \end{itemize}
    \vspace{0.5cm}
    
    \item \textbf{Frontend (Website):} Built with SvelteKit
    \begin{itemize}
        \item Uses SSR for fast loading
        \item Beautiful user interface
        \item Runs in Docker containers
    \end{itemize}
    \vspace{0.5cm}
    
    \item \textbf{Deployment:} Docker containers everywhere
    \begin{itemize}
        \item Same containers on developer's computer and server
        \item Easy to scale
        \item Easy to update
    \end{itemize}
\end{enumerate}
\end{frame}

\begin{frame}
\frametitle{SvelteKit + Go Fibers: A Real Example}

\textbf{What is Fibers?}

Fibers is a lightweight, fast web framework for Go - perfect for building APIs.

\vspace{1cm}

\textbf{How they work together:}

\begin{description}
    \item[\textbf{Go Fibers Backend}] 
    \begin{itemize}
        \item Receives requests from SvelteKit
        \item Processes business logic
        \item Talks to databases
        \item Returns data as JSON
        \item Very fast and lightweight
    \end{itemize}
    
    \vspace{0.5cm}
    
    \item[\textbf{SvelteKit Frontend}]
    \begin{itemize}
        \item Displays beautiful interface
        \item Sends user requests to Fibers backend
        \item Receives JSON data
        \item Shows updated content instantly
    \end{itemize}
\end{description}
\end{frame}

\begin{frame}
\frametitle{SvelteKit + Go Fibers Architecture}

\textbf{Communication Flow:}

\vspace{0.5cm}

\begin{enumerate}
    \item \textbf{User clicks button} in SvelteKit
    \vspace{0.2cm}
    
    \item \textbf{SvelteKit sends request} to Go Fibers API
    \begin{itemize}
        \item Example: ``Give me user data''
        \item Sent as HTTP request
    \end{itemize}
    \vspace{0.2cm}
    
    \item \textbf{Go Fibers processes} the request
    \begin{itemize}
        \item Checks permissions
        \item Queries database
    \end{itemize}
    \vspace{0.2cm}
    
    \item \textbf{Fibers sends back JSON} response
    \begin{itemize}
        \item Pure data, lightweight and fast
    \end{itemize}
    \vspace{0.2cm}
    
    \item \textbf{SvelteKit updates display} instantly
    \begin{itemize}
        \item User sees fresh data
        \item No page refresh needed
    \end{itemize}
\end{enumerate}
\end{frame}

\begin{frame}
\frametitle{Why SvelteKit + Go Fibers Works Great}

\textbf{Perfect Combination:}

\begin{columns}
\column{0.5\textwidth}
\textbf{SvelteKit Strengths}
\begin{itemize}
    \item Beautiful UIs
    \item Small bundle size
    \item Fast rendering
    \item Reactive updates
    \item SSR support
\end{itemize}

\column{0.5\textwidth}
\textbf{Go Fibers Strengths}
\begin{itemize}
    \item Lightning fast
    \item Handles many users
    \item Uses little memory
    \item Easy to write
    \item Great for APIs
\end{itemize}
\end{columns}

\vspace{1cm}

\textbf{Real-World Benefits:}
\begin{itemize}
    \item \textcolor{green}{\textbf{Speed}} - Both are optimized for performance
    \item \textcolor{green}{\textbf{Scalability}} - Can handle thousands of users
    \item \textcolor{green}{\textbf{Simplicity}} - Both use simple, clean code
    \item \textcolor{green}{\textbf{Docker Friendly}} - Both containerize perfectly
\end{itemize}
\end{frame}

\begin{frame}
\frametitle{Architecture Diagram: SvelteKit + Go Fibers}

\textbf{How the pieces fit together:}

\vspace{1cm}

\begin{center}
\textbf{User's Browser}
\begin{itemize}
    \item SvelteKit App (beautiful interface)
    \item Shows data, handles clicks
\end{itemize}

$\downarrow$ HTTP Requests (``Get my data'') $\downarrow$

\textbf{Go Fibers Server}
\begin{itemize}
    \item Processes requests
    \item Talks to database
    \item Returns JSON
\end{itemize}

$\downarrow$ JSON Responses (data) $\downarrow$

\textbf{User's Browser}
\begin{itemize}
    \item SvelteKit updates display
    \item User sees changes instantly
\end{itemize}
\end{center}

\vspace{1cm}

\textbf{All wrapped in Docker:} Same containers everywhere!
\end{frame}

\begin{frame}
\frametitle{Why This Stack Makes Sense}

\textbf{Benefits of modern development:}

\vspace{0.5cm}

\begin{itemize}
    \item \textcolor{green}{\textbf{Speed}} - Users see content fast (SSR)
    \item \textcolor{green}{\textbf{Reliability}} - Everything works the same everywhere (Docker)
    \item \textcolor{green}{\textbf{Flexibility}} - Choose the right tool for each part
    \item \textcolor{green}{\textbf{Simplicity}} - Clean languages like Go
    \item \textcolor{green}{\textbf{Performance}} - Optimized frameworks like SvelteKit
    \item \textcolor{green}{\textbf{Easy Updates}} - Change and redeploy instantly
\end{itemize}

\vspace{1cm}

\textbf{For non-technical people:}
\begin{itemize}
    \item Software works on all devices
    \item Websites load super fast
    \item Updates happen instantly
    \item No more software compatibility issues
\end{itemize}
\end{frame}

\begin{frame}
\frametitle{Key Takeaways}

\begin{enumerate}
    \item \textbf{Docker} - Like shipping containers for software. Works everywhere.
    
    \vspace{0.5cm}
    
    \item \textbf{SSR} - Server does work first, sends finished product to you. Faster loading.
    
    \vspace{0.5cm}
    
    \item \textbf{SvelteKit} - Simple, lightweight, modern framework for websites.
    
    \vspace{0.5cm}
    
    \item \textbf{Go Language} - Compiled, super-fast language. Perfect for servers.
    
    \vspace{0.5cm}
    
    \item \textbf{Go Fibers} - Fast backend API framework. Built in Go.
    
    \vspace{0.5cm}
    
    \item \textbf{Go's Speed} - Compiled code is much faster than Python or Java.
\end{enumerate}

\vspace{1cm}

\textbf{Together:} These create fast, reliable software that works everywhere.
\end{frame}

\begin{frame}
\frametitle{Questions?}

\begin{center}
    \Large
    
    Thank you!
    
    \vspace{2cm}
\end{center}
\end{frame}

\end{document}
